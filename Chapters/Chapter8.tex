\addtocontents{toc}{\vspace{2em}}
\chapter{ Συμπεράσματα και μελλοντική εργασία} % Main chapter title

\label{Chapter8}

 

\noindent Στην εργασία αυτή περιγράφεται ο σχεδιασμός και η ανάπτυξη ενός συστήματος ελέγχου για ένα βιομηχανικό σύστημα παραγωγής χρησιμοποιώντας μία προσέγγιση συνιστωσών καθώς και τεχνολογίες διαδικτύου, ώστε να μετατραπεί το παραδοσιακό βιομηχανικό σύστημα σε ένα σύστημα συμβατό με το διαδίκτυο των αντικειμένων. Η επικοινωνία των επιμέρους υποσυστημάτων έγινε με την χρήση του LwM2M που ακολουθεί της αρχές τις REST αρχιτεκτονικής και η ανάπτυξη του συστήματος ελέγχου βασίστηκε στην ανάπτυξη λογισμικού με την προσέγγιση συνιστωσών. 

\section{Συμπεράσματα}
Μέσω της εργασίας αυτής μου δόθηκε η δυνατότητα να μελετήσω αρχικά εις βάθος την προσέγγιση ανάπτυξης λογισμικού με συνιστώσες και κυρίως το framework OSGi που παρέχει αυτή τη δυνατότητα στην γλώσσα προγραμματισμού Java. Μέσω αυτής της μελέτης απέκτησα σπουδαία γνώση στον τρόπο με τον οποίο μπορεί να δομηθεί ένα πρόγραμμα ακολουθώντας την συγκεκριμένη προσέγγιση, αφού σε πολλά σημεία χρειάστηκε να ανατρέξω σε πηγαίο κώδικα προκειμένου να συλλεχθούν οι απαραίτητες πληροφορίες που χρειάστηκαν στα πλαίσια της εργασίας μου. Επιπλέον, η ομάδα που είναι υπεύθυνη για το συγκεκριμένο framework έχει κάνει άριστη δουλειά στο documentantion το οποίο ήταν σύμμαχος μου σε οποιοδήποτε πρόβλημα κλήθηκα να αντιμετωπίσω. Επιπλέον, ήρθα σε επαφή με την γλώσσα μοντελοποίησης UML, την οποία σε συνεργασία με τον επιβλέποντα καθηγητή της εργασίας κατανόησα εις βάθος ώστε να μπορέσω να σχεδιάσω το σύστημα που ανέπτυξα με τον καλύτερο δυνατό τρόπο. Ο σχεδιασμός του συστήματος με την χρήση της γλώσσας UML ήταν ίσως το σημαντικότερο μέρος της εργασίας καθώς από την στιγμή που καταλήξαμε σε μερικές παραδοχές τότε η ανάπτυξη του συστήματος ακολουθώντας τις προδιαγραφές που ορίστηκαν ήταν αρκετά δομημένη και απλή. Επίσης, ήρθα σε επαφή και μελέτησα τον τομέα του διαδικτύου των αντικειμένων. Μέσω της εργασίας κατανόησα βασικές έννοιες που τον διέπουν καθώς και προβλήματα που πρέπει να αντιμετωπιστούν ώστε οι τεχνολογίες που παρέχει αυτός ο τομέας να μπορούν να εφαρμοστούν ώστε να κάνουν τον τρόπο ζωής μας πιο απλό και εύκολο. Επιπλέον, κατανόησα βασικές έννοιες και τον τρόπο λειτουργίας των πρωτοκόλλων που χρησιμοποίησα στην παρούσα εργασία. 

	Από την εργασία προέκυψαν διάφορα συμπεράσματα, τα οποία έχουν να κάνουν κυρίως με την ανάπτυξη λογισμικού χρησιμοποιώντας την προσέγγιση συνιστωσών. Η πρώτη επαφή με την συγκεκριμένη προσέγγιση σίγουρα ήταν δύσκολη, αλλά στην συνέχεια ύστερα από αρκετή κατανόηση και έρευνα μπορούμε να καταλήξουμε στο συμπέρασμα ότι με την προϋπόθεση ότι αυτή η τεχνολογία θα χρησιμοποιηθεί σωστά τότε μπορεί να αποδώσει  σημαντικά πλεονεκτήματα στον τομέα του ΙοΤ και την αυτοματοποίησης της βιομηχανικής παραγωγής. 
	Επιπρόσθετα, το OSGi παρέχει πολλούς δωρεάν πόρους στον προγραμματιστή ώστε να κατανοήσει την λειτουργία του, τα πλεονεκτήματα που παρέχει αλλά και να βοηθηθεί κατά την διάρκεια ανάπτυξης ενός συστήματος χρησιμοποιώντας αυτή την τεχνολογία. Αρκετές υλοποιήσεις του framework ήταν ανοιχτού κώδικα και είχαμε πρόσβαση σε αυτές χωρίς κάποια επιβάρυνση. Η κοινότητα που αναπτύσσει την υλοποίηση που εμείς χρησιμοποιήσαμε έχει κάνει σημαντική δουλεία στην υλοποίηση αρκετών υπηρεσιών που παρέχει το OSGi και ήταν αρκετά βοηθητικές για την ανάπτυξη του συστήματος μας. Ωστόσο, ένα σημαντικό πρόβλημα που αντιμετωπίστηκε ήταν η διαχείριση των εξαρτήσεων των διάφορων συνιστωσών που χρησιμοποιήθηκαν στην δική μας υλοποίηση καθώς έπρεπε να ανατρέξουμε στα διάφορα μηνύματα λάθους ώστε να ανακαλύψουμε ποια components λείπουν ώστε το σύστημα να μπορέσει να λειτουργήσει. 

	Όσον αφορά το πρωτόκολλο LwM2M, αυτό με τη σειρά του προσφέρει αρκετές διευκολύνσεις. Η δύσκολη διαδικασία σχεδιασμού μιας ευέλικτης διεπαφής επάνω σε ένα πρωτόκολλο όπως το HTTP ή το CoAP, ανάγεται στη επιλογή των LwM2M αντικειμένων με τις κατάλληλες ιδιότητες. Παρ’ όλα αυτά, κατά το στάδιο του υψηλού επιπέδου σχεδιασμού του συστήματος, έγινε φανερό πως το απλό, αλλά και περιορισμένο, μοντέλο οργάνωσης των resources στο LwM2M παρουσιάζει μια αδυναμία να περιγράψει πιο σύνθετες σχέσεις μεταξύ οντοτήτων όπως για παράδειγμα η πολλαπλών επιπέδων ιεραρχία τους και η κληρονομικότητα.

	Εν τέλει, μπορεί να θεωρηθεί σίγουρο πως κάθε βιομηχανία μπορεί να επωφεληθεί σε βάθος χρόνου από την ένταξη της στο διαδίκτυο των αντικειμένων και οι προγραμματιστές στον συγκεκριμένο τομέα μπορούν να επωφεληθούν από την χρήση της προσέγγισης συνιστωσών στην ανάπτυξη του λογισμικού για τα διάφορα συστήματα. 
	
\section{Προτάσεις για μελλοντική εργασία}

Σίγουρα το σύστημα που μελετήθηκε και αναπτύχθηκε δεν είναι ικανό να επιδείξει πλήρως τα πλεονεκτήματα που παρέχει η αναβάθμιση που έγινε με την χρήση συνιστωσών. Ωστόσο κάτι τέτοιο μπορεί να γίνει με την μελλοντική μελέτη και εργασία πάνω στο συγκεκριμένο τομέα με τρόπους όπως:

\begin{itemize}
	\item{\textbf{Εύρεση λύσης στο πρόβλημα των εξαρτήσεων μεταξύ των διάφορων συνιστωσών του συστήματος}: Κατά την διάρκεια εκπόνησης της εργασίας αυτής, ένα πρόβλημα που συναντούσαμε συνέχεια μπροστά μας ήταν οι εξαρτήσεις συνιστωσών που έπρεπε να ικανοποιηθούν ώστε το σύστημα μας να είναι σε θέση να λειτουργήσει. Κάθε φορά που έπρεπε να εγκαταστήσουμε μία νέα συνιστώσα, όπως για παράδειγμα την συνιστώσα που υλοποιεί τις τεχνολογίες που ορίζει το framework για τα Declarative Services, τότε δημιουργούνταν εξαρτήσεις σε άλλες συνιστώσες και υπηρεσίες που χρειαζόταν η πρώτη συνιστώσα για να μπορέσει να ξεκινήσει η λειτουργία της. Το OSGi δεν παρείχε κάποιο μηχανισμό ώστε να αναγνωρίζονται οι επιπλέον συνιστώσες και να εγκαθίστανται μαζί με την νέα συνιστώσα που θέλαμε εμείς να εισάγουμε. Έτσι έπρεπε να γίνει μία αναζήτηση των απαιτούμενων συνιστωσών μέσα από τα μηνύματα λάθους που μας παρείχε το περιβάλλον του Felix και στην συνέχεια έπρεπε να εκαταστηθούν αυτές οι συνιστώσες που έλειπαν. Μία λύση στο συγκεκριμένο πρόβλημα θα γλύτωνε τους προγραμματιστές τέτοιων συστημάτων από πολύτιμο χρόνο.}
	\item{\textbf{Ανάπτυξη του LwM2M server χρησιμοποιώντας συνιστώσες}: Ο LwM2M server που χρησιμοποιήθηκε στην εργασία αυτή για τον έλεγχο και την επίδειξη της σωστής λειτουργίας του συστήματος μας έχει υλοποιηθεί σαν ένα POJO. Μία ενδιαφέρουσα προσέγγιση θα ήταν η χρήση συνιστωσών για την ανάπτυξη του LwM2M server. Κάτι τέτοιο θα μπορούσε να παρέχει σημαντικές νέες λειτουργίες αξιοποιώντας τα πλεονεκτήματα που παρέχει αυτή η προσέγγιση. Αν υποθέσουμε ότι η κάθε διεργασία παραγωγής κάποιου τύπου liqueur αποτελούσε ξεχωριστή συνιστώσα του συστήματος, τότε αυτή η συνιστώσα θα μπορούσε να αλλάξει και το σύστημα να είναι σε θέση να παράξει έναν νέο τύπου liqeuer χωρίς να είναι απαραίτητο να τροποποιηθεί όλο το σύστημα από την αρχή.}
	\item{\textbf{Ανάπτυξη εφαρμογής για φορητές συσκευές ώστε κάποιος να είναι σε θέση να διαχειρίζεται το σύστημα από οπουδήποτε βρίσκεται αρκεί να έχει πρόσβαση στο διαδίκτυο}: Στόχος του διαδικτύου των αντικειμένων είναι να κάνει την ανθρώπινη ζωή πιο εύκολη. Από την στιγμή που ένα τέτοιο σύστημα θα είναι σε θέση να ενταχθεί στον τομέα του διαδικτύου των αντικειμένων και σε συνεργασία με την τεράστια εξέλιξη που έχει ο τομέας των smartphones θα μπορούσε να υλοποιηθεί μία εφαρμογή μέσως της οποίας ο χρήστης θα μπορούσε να ορίσει τις προδιαγραφές του liqueur που θέλει να παραγγείλει και να ενημερώνεται με διάφορα μηνύματα για την διαδικασία παραγωγής του προϊόντος που διάλεξε. Επιπλέον, θα μπορούσε να είναι σε θέση να δει και την κατάσταση του κάθε σιλό κατά την διαδικασία παραγωγής του liqueur.}
	%\item{Βελτιώσεις στον τομέα της ασφάλειας των συστημάτων αυτών}
	\item{\textbf{Χρήση της γλώσσας UML με σκοπό την αυτόματη παραγωγή κώδικα του συστήματος μέσω καλά ορισμένων προδιαγραφών}: Κατά την διάρκεια εκπόνησης της εργασίας έγινε μία προσπάθεια αυτόματης παραγωγής του κώδικα της διεργασίας παραγωγής liqueur μέσω των διαγραμμάτων UML. Εν τέλει δεν καταφέραμε να φτάσουμε στο επιθυμητό αποτέλεσμα, αλλά το συγκεκριμένο θέμα θα μπορούσε να αποτελέσει το θέμα κάποιας άλλη έρευνας. Καθώς μέσα από τα διαγράμματα UML είναι εφικτό να περιγραφεί τόσο η δομή όσο και η συμπεριφορά ενός συστήματος, αν αυτά είναι σωστά ορισμένα και ακολουθούν συγκεκριμένουν κανόνες, τότε θα μπορούσε να παραχθεί κώδικας ο οποίος να είναι σε θέση να εκτελεστεί και να παρέχει την επιθυμητή λειτουργικότητα στο σύστημα.}
\end{itemize}