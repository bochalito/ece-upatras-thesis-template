% Chapter 1
\setlength{\parindent}{24pt}
\chapter{Εισαγωγή} % Main chapter title

\label{Chapter1} % For referencing the chapter elsewhere, use \ref{Chapter1} 

%\lhead{ΕΙΣΑΓΩΓΗ} % This is for the header on each page - perhaps a shortened title

%----------------------------------------------------------------------------------------

\section{Ιστορική αναδρομή στο Διαδίκτυο των αντικειμένων}
Η ραγδαία εξέλιξη της τεχνολογίας και κατ’ επέκταση του διαδικτύου οδηγεί τον κόσμο προς μία “συνεχώς διασυνδεδεμένη” πραγματι­κότητα.  Το διαδίκτυο βρίσκετε σχεδόν παντού, είτε ενσύρματα είτε ασύρματα, και καθημερινώς όλο και περισσότερες συσκευές συνδέονται στο διαδίκτυο. Αυτή η ανάπτυξη της συγκεκριμένης τεχνολογίας, μεγα­λώνει σε σημασία και μέσω της εξάπλωσης και των νέων τρόπων αξιοποίη­σης του δημιουργείται επιπρόσθετη αξία. Η ιστορία του διαδικτύ­ου ξεκινά με το “Διαδίκτυο των Υπολογιστών”, ένα παγκόσμιο δίκτυο που παρείχε υπηρεσίες όπως ο Παγκόσμιος Ιστός. Η ραγδαία εξέλιξη που ακο­λούθησε μας έφερε στο “Διαδίκτυο των ανθρώπων”, δημιουργώντας έτσι νέες έννοιες όπως το Κοινωνικό δίκτυο (Web 2.0) στο οποίο παράγεται περιεχόμενο από ανθρώπους ώστε να καταναλωθεί από ανθρώπους, με σύνδεση στο διαδίκτυο [1]. Σημαντικά στοιχεία που αποδεικνύουν την ρα­γδαία εξέλιξη του “διαδικτύου των ανθρώπων” είναι η τεράστια απήχηση που έχουν τα κοινωνικά δίκτυα που φιλοξενούνται στο διαδίκτυο, για παράδειγμα το Facebook με τους 2.2 δισεκατομμύρια χρήστες στο τέλος του 2017 [2]. 

Τα όρια του Διαδικτύου, διευρύνονται καθημερινά σε συνδυασμό με την τεχνολογική πρόοδο που καθιστά επιτρεπτή την πρόσβαση σε αυτό από όλο και περισσότερα σημεία με όλο και μικρότερο κόστος. Επιπλέον, η επεξεργαστική ισχύς καθώς και η χωρητικότητα των συσκευών συνε­χώς αυξάνονται αντιστρόφως ανάλογα με το μέγεθος τους. Όλο αυτό πέρα από το γεγονός ότι αλλάζει την φύση των συσκευών που οι άνθρω­ποι χρησιμοποιούν για να συνδέονται στο διαδίκτυο, δημιουργεί και σημα­ντικές νέες ευκαιρίες και εφαρμογές. Οι συσκευές αυτές απαρτίζονται από αισθητήρες και actuators ενώνοντας έτσι τον φυσικό κόσμο με τον κυβερνοχώρο. Ο συνδυασμός όλων των παραπάνω δημιουργεί μια νέα έν­νοια για το Διαδίκτυο, το “Διαδίκτυο των αντικειμένων” (Internet of Things - IoT). 

Η έννοια του Διαδικτύου των Αντικειμένων χρονολογείται από το 1982, όταν ένας αυτόματος πωλητής συνδεόταν στο διαδίκτυο για να αναφέρει τα ποτά που απομένουν καθώς και την θερμοκρασία τους. Το 1991, ένα σύγχρονο όραμα για το ΙοΤ διατυπώθηκε από τον Mark Weiser. Ωστόσο, το 1999 ο Bill Joy μας εισήγαγε στον όρο “επικοινωνία συσκευής με συσκευή” (device to device communication) [4] και τον ίδιο χρόνο ο Kevin Ashton πρότεινε τον όρο Internet of Things για να περιγράψει ένα σύστημα διασυνδεδεμένων συσκευών [5]. Τα τελευταία χρόνια, ο όρος “Διαδίκτυο των Αντικειμένων” έχει εξαπλωθεί γρήγορα. Μέχρι το 2005 εί­χε αρχίσει να εμφανίζεται σε τίτλους βιβλίων και το 2008 διεξήχθη το πρώτο επιστημονικό συνέδριο με θέμα τον συγκεκριμένο τομέα [6]. 

Το διαδίκτυο των αντικειμένων θα αλλάξει τα πάντα, ακόμα και εμάς. Το διαδίκτυο έχει αντίκτυπο στην εκπαίδευση, τις επικοινωνίες, την επιστήμη, την κυβέρνηση και την ανθρωπότητα. Είναι ξεκάθαρο ότι το ίντερνετ αποτελεί μια από τις πιο σημαντικές δημιουργίες σε όλη την αν­θρώπινη ιστορία και πλέον με την έννοια του Internet of Things μας δίνε­ται η δυνατότητα να απολαμβάνουμε μία πιο “έξυπνη” ζωή σε όλο το εύ­ρος της. Μέσω της νέας αυτής τεχνολογίας τα αντικείμενα που θα είναι συνδεδεμένα στο διαδίκτυο θα αναγνωρίζονται και θα αποκτούν συμπερι­φορά νοημοσύνης. Κάτι τέτοιο θα είναι εφικτό καθώς τα αντικείμενα θα μπορούν να επικοινωνήσουν μεταξύ τους και να ανταλλάξουν δεδομένα για την κατάσταση τους και την λειτουργία τους [7]. 
Με την εξέλιξη της τεχνολογίας του ΙοΤ, η δοκιμή και η ανάπτυξη προϊόντων θα μας φέρει πολύ κοντά στην ανάπτυξη έξυπνων περιβαλλόντων μέχρι το 2020 [8]. Στο εγγύς μέλλον η αποθήκευση και οι υπηρεσίες επικοινωνίας θα είναι πολύ διαδεδομένες. Άνθρωποι, μηχανές, έξυ­πνα αντικείμενα, ο περιβάλλων χώρος και πλατφόρμες διασυνδεδεμένες με ασύρματους ή ενσύρματους αισθητήρες, Machine to Machine (M2M) συσκευές και RFID ετικέτες θα μπορούν να δημιουργήσουν ένα δίκτυο δικτύων (network of networks) [9]. 

%----------------------------------------------------------------------------------------

\section{Αντικείμενο της εργασίας}

Αντικείμενο της παρούσας εργασία είναι η μελέτη και η αναβάθμιση παραδοσιακών εφαρμογών χρησιμοποιώντας τεχνολογίες ΙοΤ και ακολουθώντας μια προσέγγιση βασισμένη σε συνιστώσες για τον σχεδιασμό και την υλοποίηση. Πιο συγκεκριμένα, έγινε μία έρευνα στα προβλήματα που έχει να αντιμετωπίσει η βιομηχανία καθώς και στις λύσεις που παρέχει το διαδίκτυο των αντικειμένων στον συγκεκριμένο τομέα, αλλά και στις προκλήσεις που πρέπει να αντιμετωπιστούν. Στη συνέχεια, μελετήθηκαν διάφορα πρωτόκολλα επικοινωνίας των συσκευών που είναι συμβατά με το ΙοΤ και καταλήξαμε στην χρήση των CoAP και LwM2M. Για να επιτευχθεί η αναβάθμιση των παραδοσιακών εφαρμογών ακολουθήθηκε μία προσέγγιση βασισμένη σε συνιστώσες με σκοπό να μελετηθούν τα πλεονεκτήματα που μπορεί να επιφέρει η χρήση της στον συγκεκριμένο τομέα. Πιο αναλυτικά, σύμφωνα με την συγκεκριμένη προσέγγιση ένα παραδοσιακό σύστημα μπορεί να απλοποιηθεί σε επιμέρους τμήματα και κάθε τέτοιο τμήμα να αναπτυχθεί και να υλοποιηθεί ανεξάρτητα. 

Η διαδικασία αυτή στην συγκεκριμένη εργασία γίνεται μέσα από την ανάπτυξη ενός ΙοΤ-συμβατού βιομηχανικού συστήματος παραγωγής liqueur. Τα επιμέρους συστήματα αυ­τού δεν είναι χωροταξικά συγκεντρωμένα και επικοινωνούν χρησιμο­ποιώντας πρωτόκολλα επιπέδου εφαρμογής που είναι κατάλληλα για ΙοΤ εφαρμογές. Το σύστημα παραγωγής που χρησιμοποιήθηκε σαν σενάριο μελέτης στην παρούσα εργασία αποτελείται από τέσσερα βασικά σημεία επεξεργασίας της πρώτης ύλης τα οποία αποτελούνται από αισθητήρες και μηχανικά μέρη και χρησιμοποιούνται ταυτόχρονα από δύο γραμμές παραγωγής διαφορετικών προϊόντων. Κατά την διάρ­κεια της μελέτης αναπτύχθηκε ένα κατανεμημένο σύστημα ελέγχου του συστήματος παραγωγής ακολουθώντας την προσέγγιση που αναφέρθηκε παραπάνω και χρησιμοποιήθηκαν εξομοιωτές των μηχανι­κών τμημάτων του, που είχαν σχεδιαστεί και υλοποιηθεί σε προηγούμενη διπλωματική εργασία, ώστε να είναι εφικτό να γίνουν δοκιμές και έλεγχοι της σωστής λειτουργίας του συστήματος ελέγχου που υλοποιήθηκε. Η επικοινωνία μεταξύ των κατανεμημένων μικροϋπολογιστών του συστήματος βασίζεται στα πρωτόκολα CoAP και LwM2M.

Μέσα από αυτή τη μελέτη, εκτιμάται ότι η χρήση της συγκεκριμένης προσέγγισης θα προσφέρει τεράστια πλεονεκτήματα στην ανάπτυξη λογισμικού για τέτοια συστήματα, όπως για παράδειγμα λιγότερος χρόνος ανάπτυξης του λογισμικού, επαναχρησιμοποίηση ήδη υπάρχοντων συνιστωσών για την υλοποίηση ενός νέου συστήματος, αντικατάσταση συνιστωσών χωρίς την απαίτηση για διακοπή των διεργασιών του συστήματος, που περιγράφονται αναλυτικότερα στην συνέχεια. 

Από την εργασία αυτή προέκυψαν διάφορα συμπεράσματα τα οποία αφορούν κυριώς την ανάπτυξη λογισμικού χρησιμοποιώντας την προσέγγιση των συνιστωσών. Πέρα από αυτό προέκυψαν διάφορα συμπεράσματα σχετικά με την χρήση του πρωτοκόλου LwM2M για την επικοινωνία μεταξύ των συσκευών καθώς και για τα πλεονεκτήματα που αυτό μπορεί να προσφέρει στον συγκεκριμένο τομέα. Εν τέλει καταλήξαμε στο συμπέρασμα ότι κάθε βιομηχανία μπορεί να επωφεληθεί από την ένταξη της στο διαδίκτυο των αντικειμένων καθώς κάτι τέτοιο θα επιφέρει αρκετές διευκολύνσεις στον τρόπο με τον οποίο γίνεται η παραγωγή αλλά και στον τροπό με τον οποίο οι χειριστές των μηχανών αλλά και οι μηχανικοί παραγωγής θα διαχειρίζονται την παραγωγή. 

\section{Δομή της εργασίας}
Στο \textbf{κεφάλαιο 1} γίνεται μια εισαγωγή στις έννοιες του Διαδικτύου και του Διαδικτύου των αντικειμένων καθώς και στο περιεχόμενο της εργασίας και στο πρόβλημα στο οποίο έπρεπε να δωθεί λύση.

Στο \textbf{κεφάλαιο 2} ακολουθεί μια εκτενέστερη ανάλυση του Διαδικτύου των αντικειμένων καθώς και μια εξειδικευμένη ανάλυση της τεχνολογίας αυτής σε ένα συγκεκριμένο τομέα, αυτόν της βιομηχανίας. Στην συνέχεια, γίνεται μία περιγραφή της αρχιτεκτονικής REST και μία ανάλυση των δίαφορων πρωτοκόλλων που χρησιμοποιούνται για την επικοινωνία μεταξύ των συσκευών στον τομέα του διαδικτύου των αντικειμένων.
	
Στο \textbf{κεφάλαιο 3} γίνεται μια λεπτομερής περιγραφή του πρωτοκόλου LwM2M που χρησιμοποιήθηκε για την επικοινωνία των διάφορων συσκευών που απαρτίζουν το βιομηχανικό σύστημα παραγωγής. Επίσης γίνεται μία εκτενής περιγραφή των μηχανισμών που παρέχει το πρωτόκολλο αυτό ώστε να είναι εφικτή η επικοινωνία μεταξύ των συσκευών και ακολουθεί μία αναφορά στα πλεονεκτήματα που προσφέρει η χρήση του.

Στο \textbf{κεφάλαιο 4} αναλύεται η χρήση συνιστωσών στην ανάπτυξη λογισμικού, αναφέρονται οι βασικές οντότητες λογισμικού που ορίζονται από την συγκεκριμένη προσέγγιση, οι στόχοι της συγκεκριμένης προσέγγισης καθώς και τα πλεονεκτήματα που προσφέρει. Στην συνέχεια, ακολουθεί μία ανάλυση του  OSGi framework, η οποία περιλαμβάνει την αρχιτεκτονική του, τα πλεονεκτήματα που αυτό προσφέρει καθώς και μία ανάλυση των υλοποιήσεων αυτού του framework.


Στο \textbf{κεφάλαιο 5} γίνεται μια λεπτομερής περιγραφή των μηχανικών τμημάτων του συστήματος που χρησιμοποιήθηκε σαν case study καθώς και της διάταξης τους και στην συνέχεια παρατίθενται οι απαιτήσεις λειτουργίας του συστήματος παραγωγής. 

Στο \textbf{κεφάλαιο 6} περιγράφεται ο σχεδιασμός του συστήματος ελέγχου των μηχανικών υποσυστημάτων του βιομηχανικού συστήματος παραγωγής. Ο σχεδιασμός του συστήματος ελέγχου ξεκινά από τον σχεδιασμό των υπηρεσιών που πρέπει να παρέχονται στο περιβάλλον, στην συνέχεια αναλύεται ο αρχιτεκτονικός σχεδιασμός του συστήματος, η δομή του κάθε σιλό ξεχωριστά και τέλος η συμπεριφορά του συστήματος ελέγχου. 

Στο \textbf{κεφάλαιο 7} παρουσιάζεται η διαδικασία ανάπτυξης του απαραίτητου για τη λειτουργία του συστήματος λογισμικού, τα εργαλεία που χρησιμοποιήθηκαν, τα προβλήματα που αντιμετωπίστηκαν και οι λύσεις που επιλέχθηκαν. Επιπλέον, γίνεται αναφορά σε αρκετές τεχνολογίες που παρέχει το OSGi οι οποίες και χρησιμοποιήθηκαν στην υλοποίηση. 

Στο \textbf{κεφάλαιο 8} περιγράφονται τα συμπεράσματα που προέκυψαν από την παρούσα μελέτη καθώς και μερικές προτάσεις για βελτιώσεις του συστήματος.


\newpage \pagestyle{fancy} \mbox{}