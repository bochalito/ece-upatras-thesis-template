%---------------------------------------------
% Περιληψη/Ευχαριστιες
%---------------------------------------------

\abstractGr{\addtocontents{toc}{\vspace{1em}} % Add a gap in the Contents, for aesthetics

Στη σημερινή εποχή, το διαδίκτυο των Αντικειμένων αποτελεί μια εξαιρετικά σημαντική τεχνολογία που αποσκοπεί στην διευκόλυνση της ανθρώπινης ζωής, επιτρέποντας την απρόσκοπτη επικοινωνία μεταξύ αντικειμένων και μηχανών με τον άνθρωπο. Ειδικότερα στο χώρο της βιομηχανίας αντιμετωπίζεται ως η επόμενη Βιομηχανική Επανάσταση. Οι προκλήσεις που δημιουργούνται από την αναμενόμενη ένταξη ενός νέου συνόλου συσκευών στο διαδίκτυο πρέπει να αντιμετωπιστούν, με σκοπό τα οφέλη που θα αποκομίσει η ανθρώπινη κοινωνία να είναι αντάξια των υψηλών προσδοκιών από την τεχνολογία αυτή.

\hspace{1cm} 	Η παρούσα εργασία αποτελεί μια μελέτη τεχνολογιών που επιτρέπουν την επικοινωνία μεταξύ των συσκευών αλλά και  μια μελέτη σχετικά με την αναβάθμιση παραδοσιακών εφαρμογών ακολουθώντας μια προσέγγιση βασισμένη σε συνιστώσες, μέσω της ανάλυσης, του σχεδιασμού και της υλοποίησης του συστήματος ελέγχου ενός IoT-συμβατού κατανεμημένου βιομηχανικού συστήματος. Πιο συγκεκριμένα, το σύστημα ελέγχου που αναπτύχθηκε αξιοποιεί τα πρωτόκολλα CoAP και LwM2M για την επίτευξη της επικοινωνίας μεταξύ των συσκευών.

\hspace{1cm} 	Η αναβάθμιση της προυπάρχουσας εφαρμογής για τον έλεγχο του βιομηχανικού  συστήματος που μελετήθηκε, πραγματοποιήθηκε χρησιμοποιώντας μια προσέγγιση βασισμένη σε συνιστώσες και η υλοποίηση έγινε αξιοποιώντας το OSGi framework. Η αποτελεσματικότητα του προτεινόμενου συστήματος αξιολογήθηκε χρησιμοποιώντας μεταξύ άλλων και εξομοιωτές των μηχανικών μερών του φυσικού συστήματος παραγωγής. Στο τέλος γίνεται μια συνολική αξιολόγηση του συστήματος και προτείνονται κατευθύνσεις προς τις οποίες μπορεί να προσανατολιστεί η περαιτέρω ανάπτυξή του. \\[2cm]

\textbf{Λέξεις Κλειδιά:} \textit{Διαδίκτυο των Αντικειμένων, Συνιστώσα, LwM2M}

}
\newpage \pagestyle{fancy} \mbox{}
\clearpage
\pagestyle{empty}


\abstract{\addtocontents{toc}{\vspace{1em}} % Add a gap in the Contents, for aesthetics

Internet of Things is an important technology of our times,that promises to simplify everyday tasks, allowing communication between objects and machines with humans. Particularly within the industry domain, IoT has been labeled as the next Industrial Revolution. As a result, in order to unlock the full potential this new concept carries, it is essential to address the challenges that are also raised as a result.




\hspace{1cm}	The present thesis consists of a study of some technologies that allow communication between devices as well as a study on the upgrading of traditional applications following a component based approach, by analyzing, designing and implementing the control system of an IOT-compliant distributed industrial system. Specifically, the control system designed utilizes the CoAP and LwM2M protocols to achieve communication between devices.

\hspace{1cm}	The upgrade of the existing application used to control the industrial system studied was accomplished using a component-based approach and the implementation was made using OSGi framework. The effectiveness of the system designed is measured using simulators of the mechanical parts the production system consists of. Finally, a review of the system and the development procedure is given, along with suggestions for future development. \\[2cm]

\textbf{Keywords:} \textit{Internet of Things, Component, LwM2M}


}
\newpage \pagestyle{fancy} \mbox{}
\clearpage
\pagestyle{empty}

\acknowledgements{\addtocontents{toc}{\vspace{1em}} % Add a gap in the Contents, for aesthetics
\begin{flushright}
 \textit{{"Scientists dream of great things. Engineers do them."}}

 {James Michener, Author}
\end{flushright}
Αρχικά θα ήθελα να ευχαριστήσω θερμά τον επιβλέποντα καθηγητή κύριο Κλεάνθη Θραμπουλίδη τόσο για την ευκαιρία που μου έδωσε να ασχοληθώ με το ιδιαίτερα ενδιαφέρον αντικείμενο της αναβάθμισης ενός παραδοσιακού βιομηχανικού συστήματος χρησιμοποιώντας την προσέγγιση των συνιστωσών και τεχνολογίες διαδικτύου με σκοπό αυτό το σύστημα να ενταχθεί στο διαδίκτυο των αντικειμένων όσο και για την στήριξη και την πολύτιμη καθοδήγηση που μου πρόσφερε σε όλα τα στάδια της προσπάθειας αυτής. Οι γνώσεις και η εμπειρία που απέκτησα κατά την εκπόνηση της διπλωματικής εργασίας αυτής αποτελούν εξαιρετικά εφόδια για την μετέπειτα πορεία μου.

\hspace{1cm}	Επίσης θα ήθελα να ευχαριστήσω θερμά τον Μπουλούμπαση Ιωάννη για την πολύτιμη βοήθεια του σε διάφορα προβλήματα που αντιμετωπίσα καθώς και για την υλοποίηση του LwM2M server που χρησιμοποιήσα για τον έλεγχο της λειτουργικότητας του δικού μου συστήματος.

\hspace{1cm}	Τέλος θα ήθελα να ευχαριστήσω θερμά την οικογένεια μου για την αμέριστη συμπαράσταση και κατανόηση που έδειξε σε όλη την διάρκεια της πανεπιστημιακής μου πορείας.

}

\newpage \pagestyle{fancy} \mbox{}
\clearpage
\pagestyle{empty}

